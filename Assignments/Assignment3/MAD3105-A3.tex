%%%%%%%%%%%%%%%%%%%%%%%%%%%%%%%%%%%%%%%%%
% Short Sectioned Assignment
% LaTeX Template
% Version 1.0 (5/5/12)
%
% This template has been downloaded from:
% http://www.LaTeXTemplates.com
%
% Original author:
% Frits Wenneker (http://www.howtotex.com)
%
% License:
% CC BY-NC-SA 3.0 (http://creativecommons.org/licenses/by-nc-sa/3.0/)
%
%%%%%%%%%%%%%%%%%%%%%%%%%%%%%%%%%%%%%%%%%

%----------------------------------------------------------------------------------------
%	PACKAGES AND OTHER DOCUMENT CONFIGURATIONS
%----------------------------------------------------------------------------------------

\documentclass[paper=a4, fontsize=11pt]{scrartcl} % A4 paper and 11pt font size

\usepackage[T1]{fontenc} % Use 8-bit encoding that has 256 glyphs
\usepackage{fourier} % Use the Adobe Utopia font for the document - comment this line to return to the LaTeX default
\usepackage[english]{babel} % English language/hyphenation
\usepackage{amsmath,amsfonts,amsthm, pgf, tikz} % Math packages
\usetikzlibrary{arrows, automata}
\usepackage{fancyhdr}
\usepackage{listings}
\usepackage{color}
\usepackage{lipsum} % Used for inserting dummy 'Lorem ipsum' text into the template

\usepackage{sectsty} % Allows customizing section commands
\allsectionsfont{\centering \normalfont\scshape} % Make all sections centered, the default font and small caps

\usepackage{fancyhdr} % Custom headers and footers
\pagestyle{fancyplain} % Makes all pages in the document conform to the custom headers and footers
\fancyhf{}
\fancyhead[R]{Tyler O. Moses}
\fancyhead[L]{MDA 3105}
\fancyhead[C]{Discrete Mathematics II}
\fancyfoot[L]{} % Empty left footer
\fancyfoot[C]{} % Empty center footer
\fancyfoot[R]{\thepage} % Page numbering for right footer
\renewcommand{\headrulewidth}{0pt} % Remove header underlines
\renewcommand{\footrulewidth}{0pt} % Remove footer underlines
\setlength{\headheight}{13.6pt} % Customize the height of the header

\numberwithin{equation}{section} % Number equations within sections (i.e. 1.1, 1.2, 2.1, 2.2 instead of 1, 2, 3, 4)
\numberwithin{figure}{section} % Number figures within sections (i.e. 1.1, 1.2, 2.1, 2.2 instead of 1, 2, 3, 4)
\numberwithin{table}{section} % Number tables within sections (i.e. 1.1, 1.2, 2.1, 2.2 instead of 1, 2, 3, 4)

\setlength\parindent{0pt} % Removes all indentation from paragraphs - comment this line for an assignment with lots of text

\definecolor{dkgreen}{rgb}{0,0.6,0}
\definecolor{gray}{rgb}{0.5,0.5,0.5}
\definecolor{mauve}{rgb}{0.58,0,0.82}

\lstset{frame=tb,
  language=Java,
  aboveskip=3mm,
  belowskip=3mm,
  showstringspaces=false,
  columns=flexible,
  basicstyle={\small\ttfamily},
  numbers=none,
  numberstyle=\tiny\color{gray},
  keywordstyle=\color{blue},
  commentstyle=\color{dkgreen},
  stringstyle=\color{mauve},
  breaklines=true,
  breakatwhitespace=true,
  tabsize=3
}

%----------------------------------------------------------------------------------------
%	TITLE SECTION
%----------------------------------------------------------------------------------------

\newcommand{\horrule}[1]{\rule{\linewidth}{#1}} % Create horizontal rule command with 1 argument of height

\title{	
\normalfont \normalsize 
\textsc{Florida State University} \\ [25pt] % Your university, school and/or department name(s)
\horrule{0.5pt} \\[0.4cm] % Thin top horizontal rule
\huge Assignment 3: Equivalence Relations \\ % The assignment title
\horrule{2pt} \\[0.5cm] % Thick bottom horizontal rule
}

\author{Tyler Moses} % Your name

\date{\normalsize January 31, 2018} % Today's date or a custom date

\begin{document}

\maketitle % Print the title

%----------------------------------------------------------------------------------------
%	PROBLEM 1
%----------------------------------------------------------------------------------------

\section{Exercise}

Determine whether or not the below relations on the set $A = \{0,1,2,3\}$ are equivalence relations. If not, state all missing properties.

$$R = \{(0,0),(1,1),(2,2),(2,1),(1,2)\}$$
$$R = \{(3,1),(2,1),(1,2),(0,0),(2,2),(3,3),(1,1)\}$$
$$R = \{(0,0),(1,1),(2,2),(3,3)\}$$

\subsection{Solution}

\textbf{(a)}

$$R = \{(0,0),(1,1),(2,2),(2,1),(1,2)\}$$

R is symmetric and transitive but not reflexive due to the fact that $(0,0) \not\in R$.

\textbf{(b)}

$$R = \{(3,1),(2,1),(1,2),(0,0),(2,2),(3,3),(1,1)\}$$

R is reflexive but not symmetric due to the fact that $(3,1) \in R$ but $(1,3) \not\in R$. Nor is it transitive because 
$(3,1) \in R \land (1,2) \in R$ but $(3,2) \not\in R$.

\textbf{(c)}

$$R = \{(0,0),(1,1),(2,2),(3,3)\}$$

R is reflexive, symmetric, and transitive because it is the identity relation on A.

%----------------------------------------------------------------------------------------
%	PROBLEM 2
%----------------------------------------------------------------------------------------

\section{Exercise}

Determine whether or not the below relations are equivalnce relations, if not state all properties.

$$M_R = \begin{bmatrix}
1 & 0 & 1 \\
1 & 0 & 1 \\
1 & 1 & 1 \end{bmatrix}$$

$$M_R = \begin{bmatrix}
1 & 1 & 1 \\
1 & 0 & 0 \\
1 & 0 & 1 \end{bmatrix}$$

\subsection{Solution}

For each of these let $M_R$ be treated as a zero indexed matrix as in most programming languages.

\textbf{(a)}

$$M_R = \begin{bmatrix}
1 & 0 & 1 \\
1 & 0 & 1 \\
1 & 1 & 1 \end{bmatrix}$$

$M_R$ is not reflexive due to the fact that $M_R[1][1] = 0$.

$M_R$ is not symmetric due to the fact that $M_R[1][0] = 1 \land M_R[0][1] = 0$.

$M_R$ is not transitive due to the fact that $M_R[1][2] = 1 \land M_R[2][1] = 1 \land M_R[1][1] = 0$.

\textbf{(b)}

$$M_R = \begin{bmatrix}
1 & 1 & 1 \\
1 & 0 & 0 \\
1 & 0 & 1 \end{bmatrix}$$

$M_R$ is not reflexive due to the fact that $M_R[1][1] = 0$.

$M_R$ is not transitive due to the fact that $M_R[1][0] = 1 \land M_R[0][1] = 1 \land M_R[1][1] = 0$.

$M_R$ is symmetric.

%----------------------------------------------------------------------------------------
%	PROBLEM 3
%----------------------------------------------------------------------------------------

\section{Exercise}

What are the equivalence classes of the below bit strings for the equivalence
relation, R, on the set of all bit strings of length 2 or more, where R consists of all
pairs $(x, y)$ such that x and y are bit strings of length 2 or more that agree (are the
same) except possibly (i.e. they may differ) in their first 2 bits. (You may describe the
elements in the sets instead of listing them all out).

(a) 1001 (b) 10 (c) 10111 

\subsection{Solution}

\textbf{(a) 1001} 

$$ \{1001\} = \{ab01 \vert \forall a,b \in \{0,1\}\}$$

\textbf{(b) 10} 

$$ \{10\} = \{ab \vert \forall a,b \in \{0,1\}\}$$

\textbf{(c) 10111} 

$$ \{10111\} = \{ab111 \vert \forall a,b \in \{0,1\}\}$$

%----------------------------------------------------------------------------------------
%	PROBLEM 4
%----------------------------------------------------------------------------------------

\section{Exercise}

4) Which of these collections of subsets are partitions of $A = \{a,b,c,d,e\}$? If not, why?

(a) $\{a,b\},\{c\},\{d\}$

(b) $\{a\},\{b,d\},\{c\},\emptyset$

(c) $\{a,b,d\},\{c,d\},\{e\}$

\subsection{Solution}

\textbf{(a)}

$$\{a,b\},\{c\},\{d\}$$

While the subsets might be disjoint and non empty, the union of them is not A.

\textbf{(b)}

$$\{a\},\{b,d\},\{c\},\emptyset$$

They are disjoint, but one of them is empty and their union is also not A.

\textbf{(c)}

$$\{a,b,d\},\{c,d\},\{e\}$$

This time, they are empty and their union is A, but $d$ is in two sets making them not disjoint.


%----------------------------------------------------------------------------------------
%	PROBLEM 5
%----------------------------------------------------------------------------------------

\section{Exercise}

List the ordered pairs in the equivalence relations produced by the following
partitions of $A = \{a,b,c\}$.

(a) $\{\{a\},\{b\},\{c\}\}$

(b) $\{\{c\},\{a,b\}\}$

\subsection{Solution}

\textbf{(a)}

$$\{\{a\},\{b\},\{c\}\}$$

This is the identity relation:

$$\{(a,a),(b,b),(c,c)\}$$

\textbf{(b)}

$$\{\{c\},\{a,b\}\}$$

This is not the identity relation but close:

$$\{(a,a),(b,b)(a,b),(b,a),(c,c)\}$$


%----------------------------------------------------------------------------------------
%	END DOCUMENT
%----------------------------------------------------------------------------------------

\end{document}